\documentclass[12pt]{article}
\setlength{\textheight}{9.80in}
\setlength{\textwidth}{6.40in}
\setlength{\oddsidemargin}{0.0mm}
\setlength{\evensidemargin}{1.0mm}
\setlength{\topmargin}{-0.6in}
\setlength{\parindent}{0.2in}
\setlength{\parskip}{1.5ex}
\newtheorem{defn}{Definition}
\renewcommand{\baselinestretch}{1.2}

\begin{document}

\thispagestyle{empty}

\bibliographystyle{prsty}


\title{The Landscape Used For the  Flight Program}

\author{C. Godsalve \\
   email: seagods@hotmail.com}

\maketitle

The basis of the landscape used in the ``Flight" program is a triangular network.
Triangular networks may be regular or irregular, the Triangular Irregular Network (TIN) for instance,
 is a regular workhorse for Geographic 
 Information Systems (GIS)\cite{GIS:Mybib}. It just consists of a set of points or {\it nodes}
which are connected together so that a given region of the plane is covered by a mesh of
triangles. For such a network to be of practical use in GIS (for instance) there
are certain rules about the triangulation. Pretty well,  they are demonstrated
in Fig.1 where we see three basic rules.
\begin{figure}[htb]
\vspace*{10cm}
\special{psfile=Rules.eps vscale=50 hscale=50 voffset=10 hoffset=80}
\caption{ A; triangles cannot overlap: B; nodes cannot occur on edges: C; triangles must have the same orientation.}
\end{figure}
In Fig.1 part C, what do we mean by  the ``orientation" of a triangle? 
Naturally, each node is given a number, and each triangle has a number, and in 
any triangle three nodes must be listed {\it in order}. 

In the simple case of
a single valued surface defined as heights over each node in the plane, such as occurs in many landscapes,
 this order is defined by whether
the listing of the nodes 1, 2 and 3 in each triangle  appears as clockwise or anti-clockwise from a given side
of that plane.
The nodes listed in the triangles must {\it all} be either 
anticlockwise or clockwise. 
Quite often, the nodes are not regularly positioned in any way or order. 
In such cases, a {Delaunay triangulation}\cite{Delaunay:Mybib} is usually used to go from
 a bunch of nodes on a plane to a triangulation. 

In this article, the ``landscape" generated is defined by a square array of nodes
and the network is regular. Such landscapes may be thought of as a special subset of
possible TINs.
To generate a square consisting of $N\times N$ lesser squares, we need a
square array of $(N+1)\times(N+1)$ nodes, and at each node, the  height above the
plane in which the nodes exist, defines our ``landscape". (Heights can, of course, be negative.) 
If we wish our landscape to consist of simple flat facets, we 
must make the facets triangular. The reason is simply that, in general, four points in a three dimensional space
 are not generally co-planar. However, three
points in a three dimensional space define a plane (unless they lie in a straight line or coincide with each other).

So, for any square consisting of $N\times N$ small(est) squares, each small square 
can be split into two triangles. Unfortunately, without extra information on a sub-grid scale, this 
does not uniquely specify the landscape as seen in Fig.2. Suppose we denote $H$ as (relatively) high, and
 $L$ as relatively low, we cannot determine whether we have a square with a ``valley" along one diagonal
of the square or a ``ridge" on the other diagonal.
\begin{figure}[htb]
\vspace*{10cm}
\special{psfile=Decide.eps vscale=50 hscale=50 voffset=10 hoffset=80}
\caption{ Without extra information we can't decide to put in a valley between nodes 2 and 4 or a ridge between 
nodes 1 and 3.}
\end{figure}

We can split each square into two triangles along either diagonal as we please. For a realistic attempt at 
representing the landscape we can use the given height data, and with those cases which are undecidable, we may
use data from the eight surrounding squares to decide the best choice to make as to which 
diagonal defines the two triangles (given assumptions on the nature of
the landscape.) 

In the program {\it fractalsurf} the two sub-triangles of the grid squares are defined as in Fig.3.
That is, we do not have sub-grid information, and ignore any niceties which may be used to infer cases
where things cannot be decided from the data at the corners of a given square. We have called the two ways
to split a square ``even" and ``odd" and used a chequerboard pattern to ``break things up". (All odd or all even
would tend to introduce more artifacts than otherwise, another choice would be to choose odd and even at random.)

So, we wish to generate a landscape. To generate a ``skyline" or 1D landscape  we can use random midpoint
 displacement\cite{Fractals:Mybib}.
Here we may start with two values at the end of a line segment. We may take the mean, and then generate
a pseudo random number from a Gaussian distribution of some variance and add it to that mean to get the
value of the ``skyline" at the midpoint. The end points, tegether with the new midpoint, makes two new
 line segments.
We may now take the two line segments
and generate the midpoint values at two new midpoints. We halve the variance at this stage. Each time
we end up with double the number of line segments, and we halve the variance every time. We might add
 a random number chosen in the same way to the set of values we are taking the midpoints of.
This, and many other kinds of algorithms for generating fractal ``skylines", ``landscapes", and ``clouds"
are discussed extensively with many examples along with pseudo-code, are given in \cite{Fractals:Mybib}.

 The ``fractalsurf" program uses a variation of the random midpoint displacement in 2D. The triangles belong to 
a triangle class. As well as three node numbers, each triangle has three neighbours.
 If $m$=1, 2, or 3 are the nodes, then edge $m$ is the edge {\it not} containing node $m$,
and neighbour $m$ is the neighbouring triangle which lies over edge $m$. Neighbour encoding was
 to facilitate
 geodesic propagation across the landscape at a later stage in code development. In the ``Flight program", we generate
a linked list of the triangles attached to each node in the mesh. This is used to generate
the normals required at each node if Gouraud shading is to be used in conjunction with OpenGL lighting.
It's easy enough to do this without the linked lists because of the regularity of the mesh, but it is
included in case irregular meshes are used at a later stage in code development.


\begin{figure}[htb]
\vspace*{10cm}
\special{psfile=Subgrid.eps vscale=50 hscale=50 voffset=10 hoffset=80}
\caption{ We assign which squares are odd or even in a regular chequerboard pattern without considering
 how to decide in cases such as Fig.2.
}
\end{figure}



 


\bibliography{/home/daddio/Articles/Mybib}

\end{document}

